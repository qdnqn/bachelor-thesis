\begin{samepage}
 \chapter*{Sažetak}$~$

Rad se bavi problemom automatizacije korištenjem novih tehnologija i novih načina deployment-a aplikacija. Fokus je stavljen na korištenje Ansible-a kao pomoćnog alata pri automatizaciji procesa kreiranja okruženja i mreže, te automatizacije samog procesa kreiranja artifakata i deployment-a istih na klasterizovano okruženje. Cilj je bio istražiti kako se Ansible može integrirati u automatizaciju navedenih procesa.
\\\\
Da bi vršili automatizaciju moramo poznavati manualne korake koji se već odrađuju da bi se došlo do krajnjeg cilja. Istraživanje je pokrilo i skup tehonologija kao što su: Docker, Kubernetes, Vagrant i Maven. Načinjena je cjelina u pogledu automatizacije dostavljanja software-a od nule do kreirane kompletne infrastrukture spremne da pokrene kontejnerizovane aplikacije.
\\\\
Ansible se pokazao kao veoma dobar alat za provizioniranje okruženja, za šta se i najčešće koristi. Pri procesu razvoja software-a i automatizacije deployment-a također je moguće inkorporirati Ansible i iskoristiti dobre strane alata. Međutim u tom kontekstu se ne koristi često i postoje drugi alati koji preuzimaju ulogu u tom smislu.
\\\\
\vspace{10mm}
\setlength{\parindent}{0pt}
\textbf{Ključne riječi:} \textit{Automatizacija, Ansible, Maven, Docker, Kuberentes, Mreža.}
\end{samepage}
\addcontentsline{toc}{chapter}{Sažetak}